\section{Big Data Use Cases}\label{S:usecases}

The NIST Big Data Working group has identified 51 benchmarking examples for Big Data\cite{??} spanning application areas such as:

\begin{description}

\item[Government Operation:] National Archives and Records
  Administration, Census Bureau

\item[Commercial:] Finance in Cloud, Cloud Backup, Mendeley
  (Citations), Netflix, Web Search, Digital Materials, Cargo shipping
  (as in UPS)

\item[Defense:] Sensors, Image surveillance, Situation Assessment
  Healthcare

\item[Life Sciences:] Medical records, Graph and Probabilistic
  analysis, Pathology, Bioimaging, Genomics, Epidemiology, People
  Activity models, BiodiversityDeep Learning and Social Media: Driving
  Car, Geolocate images/cameras, Twitter, Crowd Sourcing, Network
  Science, NIST benchmark datasets

\item[The Ecosystem for Research:] Metadata, Collaboration, Language
  Translation, Light source experiments

\item[Astronomy and Physics:] Sky Surveys compared to simulation,
  Large Hadron Collider at CERN, Belle Accelerator II in JapanEarth,

\item[Environmental and Polar Science:] Radar Scattering in
  Atmosphere, Earthquake, Ocean, Earth Observation, Ice sheet Radar
  scattering, Earth radar mapping, Climate simulation datasets,
  Atmospheric turbulence identification, Subsurface Biogeochemistry
  (microbes to watersheds), AmeriFlux and FLUXNET gas sensors

\item[Energy:] Smart grid

\end{description}

In addition we have 81 student projects from classes taught at Indiana University on the topic of Big Data. 
From these examples we have examined on six of the NIST uses case projects to identify the technologies used as shown in Table~\ref{T:usecases}.

\begin{table}[htb]
  \bigskip
  \setlength\tabcolsep{2pt}
  \caption{
    Technology used in a subset of usecases. A \OK~indicates that the technology is used in the given project. See Table~\ref{T:usecase2} for details on a specific project. The final row aggregates \OK~across projects.}
  \label{T:usecases}
  \bigskip
  \begin{small}
    \begin{center}
      \resizebox{\columnwidth}{!}{
        \begin{tabular}{|c|*{23}c|}
          \hline

          ID & \rot{Hadoop} & \rot{Mesos} & \rot{Spark} & \rot{Storm} & \rot{Pig} & \rot{Hive} & \rot{Drill} & \rot{HBase} 
             & \rot{Mysql} & \rot{MongoDB} & \rot{Mahout} & \rot{D3 and Tableau} & \rot{nltk} & \rot{MLlib} 
             & \rot{Lucene/Solr} & \rot{OpenCV} & \rot{Python} & \rot{Java} & \rot{Ganglia} & \rot{Nagios} & \rot{zookeeper} & \rot{AlchemyAPI}
             & \rot{R} \\ \hline

          % ID  & Hadoop & Mesos & Spark & Storm & Pig & Hive & Drill & HBase & msql & mongo & mht & D3/T & nltt & mllib & lu/slr & OCV & Py  & Java & Gngla & Ngs & ZK  & AlchemyAPI & R   
          N$_1$ & \OK    &       & \OK   &       &     & \OK  & \OK   & \OK   & \OK  &       &     &      &      &       &        &     &     & \OK  & \OK   & \OK & \OK &            &     \\ \hline
          N$_2$ &        & \OK   & \OK   &       &     &      &       &       &      &       &     &      &      &       &        & \OK & \OK &      &       &     & \OK &            &     \\ \hline
          N$_3$ &        &       &       & \OK   &     &      &       & \OK   &      & \OK   &     & \OK  & \OK  &       &        &     & \OK & \OK  &       &     & \OK & \OK        & \OK \\ \hline
          N$_4$ & \OK    &       & \OK   &       &     &      &       & \OK   &      &       & \OK & \OK  &      & \OK   & \OK    &     &     & \OK  &       &     & \OK &            &     \\ \hline
          N$_5$ & \OK    &       & \OK   &       &     &      &       &       &      &       & \OK & \OK  &      & \OK   &        &     &     & \OK  &       &     &     &            &     \\ \hline
          N$_6$ & \OK    &       & \OK   &       & \OK & \OK  &       & \OK   &      & \OK   & \OK & \OK  &      & \OK   & \OK    &     &     & \OK  &       &     & \OK &            &     \\ \hline
          count & 4      & 1     & 5     & 1     & 1   & 2    & 1     & 4     & 1    & 2     & 3   & 4    & 1    & 3     & 2      & 1   & 2   & 5    & 1     & 1   & 5   & 1          & 1   \\ \hline

        \end{tabular}
      }
    \end{center}
  \end{small}
\end{table}


\begin{table*}[htb]
  \caption{Dataset used in the various use cases.}
  \bigskip
  \label{T:usecase2}
  \begin{center}
    \begin{tabular}{|c|p{0.8\columnwidth}|p{0.8\columnwidth}|c|}

      \hline

      ID    & Use Case                                                                & Dataset                                                                      & Size (GB) \tabularnewline \hline
      N$_1$ & Fingerprint Matching                                                    & Special Database 14 - NIST Mated Fingerprint Card Pairs 2                    & 2.1       \tabularnewline \hline
      N$_2$ & NIST Human and Face Detection                                           & INRIA Person Dataset                                                         & 0.96      \tabularnewline \hline
      N$_3$ & NIST Twitter Analysis                                                   & Twitter                                                                      & -         \tabularnewline \hline
      N$_4$ & NIST Analytics for Healtcare Data / Health Informatics                  & Medicare Part-B in 2014 from Center for Medicare and Medicaid Services (CMS) & 0.1       \tabularnewline \hline
      N$_5$ & NIST Spatial Big Data/Spatial Statistics/Geographic Information Systems & Uber                                                                         & 0.2       \tabularnewline \hline
      N$_6$ & NIST Data Warehousing and Data Mining                                   & United States 2010 Census data                                               & -         \tabularnewline \hline

    \end{tabular}
  \end{center}
\end{table*}


Deployment of a single technology may result in development of several ansible roles comprising dependencies.
The implementation of the Fingerprint usecases (N$_1$) Ansible playbook uses 19 such roles that may be reused in other projects and Face Detection (N$_2$) uses five.
Furthermore, The remaining four NIST use cases N$_4$ - N$_6$ (Twitter analysis, Healthcare, Spatial data, Data Wharehousing) may contribute an addition 27 roles.

In summary, we have looked at 6 projects from the 51 NIST use cases to identify 51 Ansible roles.
Looking through 81 class projects over two semesters at Indiana University showed 62 roles, of which a subset were found to be repeatedly used across various projects.

% In addition we analyzed 36 projects from a big data class taught at
% Indiana University in Spring '16. Here we identified 41 roles in 45
% projects showcasing a wide divergent use case scenario.  A similar
% class thought in Fall '15 resulted in 62 roles while using 39
% datasets.




\subsection{Fingerprint Matching (N$_1$)}

Fingerprint recognition refers to the automated method for verifying a
match between two fingerprints and that is used to identify
individuals and verify their identity. Fingerprints are the most
widely used form of biometric used to identify individuals. The
automated fingerprint matching generally required the detection of
different fingerprint features (aggregate characteristics of ridges,
and minutia points) and then the use of fingerprint matching
algorithm, which can do both one-to- one and one-to- many matching
operations. Based on the number of matches a proximity score (distance
or similarity) can be calculated. Furthermore, NIST is providing via
the the NIST Fingerprint dataset a special database. The goal for this
usecase is the following: given a set of {\it probe} and {\it gallery}
images, compare the probe images to the gallery images, and report the
matching scores.  The dataset comprises 54,000 images and their
metadata.  It uses MINDTCT \cite{mindtct} preprocesses the images
to identify minutae of the prints automatically locating and recording
ridge ending and bifurcations in a fingerprint image; and BOZORTH3
\cite{garris2001user} to identify matches. Both are are part of the NIST
Biometric Image Software (NBIS) \cite{watson2007user}.

To execute this usecase\cite{nist-fingerprint} we need to deploy an
application to analyze the dataset. It internally uses cloudmesh to
deploy an the software stack The implemented \cite{nist-fingerprint}
solution uses a software stack comprising of Hadoop HDFS\cite{hadoop},
YARN\cite{hadoop}, Apache Spark\cite{apache-spark}, Apache
HBase\cite{apache-hbase}, and Apache drill\cite{apache-drill},
Scala\cite{scala-lang}, and the NBIS
software\cite{flanagan2010nist}. A Hadoop cluster is deployed and YARN
used to schedule Spark jobs that load the images into HBase, process
the images, and compute the matches. Apache Drill, with the HBase
plugin, can then be used to generate reports with the NBIS
tools\cite{watson2007user}. The results are stored in HBase and Apache
Drill is used to query the results.  The code leverages tools and
services is based on \cite{nist-bd-pwg}, while significantly enhancing
it with cloudmesh deployment strategies and services.


\subsection{Face Detection (N$_2$)}

Human detection and face detection have been studied during the last
several years and models for them have improved along with Histograms
of Oriented Gradients (HOG) \cite{dalal2005histograms} for Human
Detection.



We use\cite{nist-facedetection} OpenCV \cite{bradski2000opencv}, a
Computer Vision library including the Support Vector Machine (SVM)
classifier, and the Histogram of Oriented Gradient (HOG)
\cite{dalal2005histograms}object detector for pedestrian detection and
INRIA Person Dataset is one of popular samples for both training and
testing purposes. HOG with SVM model is used used as object detectors
and classifiers while the python libraries from OpenCV provide these
models for human detection.  The OpenCV Python code runs with Spark
Map function to perform distributed job processing on the Mesos
scheduler.

To enable this analysis we use cloudmesh to deployed Apache Spark on a
Mesos clusters and install the OpenCV software and its Python API. We
also update the python software stack. Then we to train and apply
detection models from OpenCV using Python API. We use the INRIA Person
Dataset \cite{dalal2005inria}. This dataset contains positive and
negative images for training and test purposes with annotation files
for upright persons in each image. 288 positive test images, 453
negative test images, 614 positive training images and 1218 negative
training images are included along with normalized 64x128 pixel
formats. The size of the dataset is 970MB.

Cloudmesh deploys and builds the clusters for batch-processing large
datasets, Internally cloudmesh uses for this ansible scripts to
support installation and configuration while leveraging available
cloud compute resources. We have for this example developed or are
reusing five ansible roles that we developed for other usecases::
Apache Spark Role \cite{ansible-role-spark} Apache Mesos Mesos
\cite{hindman2011mesos}, Apache Zookeeper Role
\cite{hunt2010zookeeper}, OpenCV Role (with Python)
\cite{ansible-role-opencv}.


