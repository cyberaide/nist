

\section{Background}

\subsection{Deployment}

Big Data Software Stack is built with various software packages from different layers (and it’s increasing), therefore accomplishing these multiple software deployment is challenging



\section{Ansible}


\subsection{Why Ansible?}
Expandable: Leverages Python (default) but modules can be written in any language
Simplicity: 
Agentless: no setup required on managed node
Security: Allows deployment from user space; uses ssh for authentication
Flexibility: only requires ssh access to privileged user
Transparency: YAML Based script files express the steps of installing and configuring software
Modularity: Single Ansible Role (should) contain all required commands and variables to deploy software package independently
Sharing and portability: roles are available from source (github, bitbucket, gitlab, etc) or the Ansible Galaxy portal


\subsection{Features of Ansible fostering reproducibility}

Ansible provides a structured environment for defined sets of tasks to execute on a number of nodes. Tasks are ideally idempotent.
There are two basic methods for organizing tasks:
Playbooks: main entry points, specific to environments
Roles: reusable abstraction (analogous to functions)
Playbooks provide the primary entry point, and consist of tasks that are either defined directly in the playbook, or externally via roles.
Roles are groups of related tasks that can be distributed and reused independently of playbooks.
The Inventory file identifies the nodes controlled by Ansible.


\subsection{Ansible: Inventory, IaaS, PaaS}

The inventory file for Ansible groups the IP addresses of the infrastructure into named sets.
On the right are two examples of inventory files:
Top: a static inventory for a cluster with two nodes in the "webservers" group and one node in the "databases" group.
Bottom: a dynamic inventory in which a script is called to define the nodes. The output must be specifically-formatted JSON for Ansible.
\begin{Verbatim}
[webservers]
10.0.0.1
10.0.0.2
[databases]
10.0.0.3
#!/use/bin/env bash

python dynamic-nodes-setup.py
\end{Verbatim}

\subsection{Ansible: Roles}
Roles are primarily defined as a collection of related tasks to accomplish a single goal.  These tasks can be configured via variables, that may be overridden when the role is used in a playbook.
On the top right is an example directory structure defining a role for Apache Spark: Tasks are defined in the tasks/main.yml file, Variables are defined in defaults/main.yml
On bottom right is an elided set of the entries in tasks/main.yml. These tasks use the ``get\_url'', and ``unarchive'' modules and optionally builds from source when spark\_package\_type is "src"
\begin{Verbatim}
spark
- defaults
   - main.yml
- LICENSE
- README.md
- tasks
    - build.yml
    - main.yml
\end{Verbatim}

\begin{Verbatim}
---

- name: download
  get_url:
    url: "{{ spark_url }}"
    dest: "{{ ansible_env.PWD }}/{{ _spark_package_name }}"
    sha256sum: "{{ spark_sha256 }}"
  
- name: extract
  unarchive:
    dest: "{{ spark_install_dest }}"
    src: "{{ ansible_env.PWD }}/{{ _spark_package_name }}"
    copy: no
    creates: "{{ spark_install_dest }}/{{ _spark_directory }}"
  
- include: build.yml
  when: spark_package_type == 'src'

# ...

Ansible: Playbooks
An example playbook is shown to the right.

Name: description of this play

Hosts: the set of nodes on which to operate, named in the inventory file

Become: escalate privileges

Roles: list of roles to apply, with overridden variables as needed.
---

- name: install and configure spark
  hosts: frontendnodes
  become: yes
  roles:
    - role: java
    - role: maven
    - role: spark
      spark_version: 1.6.0
      spark_package_type: bin
\end{Verbatim}


\subsection{Ansible: Role Repositories}

There are two primary mechanisms available to redistribute roles

Ansible Galaxy is the primary hub for sharing roles. The
ansible-galaxy tool can install these roles. Useful for using
externally-developed roles

Directly from the source code repositories. In the case of using Git,
these repositories can be installed as git submodules. Useful for
internally-developed roles

We have found that installing roles as git submodules allows the
flexibility to both use roles developed elsewhere as well as make
modifications and even contribute those changes upstream.

\subsection{Ansible: Vetting Roles}
An important aspect of developing roles is the vetting step. In a
production environment, each role must be examined for security
vulnerabilities as well as to ensure that they will function as
expected. Externally-developed roles may need to be adapted to work in
the target environment. When a role is defined, variables to modify
the deployment behavior need to be exposed. We have found that many
roles made available only expose a subset of the application's
available configuration options and may not work outside the author's
environment.

Projects
List of Projects
27 roles are collected from 6 NIST use cases. Hadoop, Spark, HBase,
D3, Tableau, Java and Zookeeper are frequently reused more than four
out of six use cases.

Table 1. List of Projects with Roles

Table is in Google sheet: 
% \url{https://docs.google.com/spreadsheets/d/1idB11ibeITgTLUT9D_xT6cOMsVzdG2cVa3o0w_Wh7IE/edit?usp=sharing}

Roles from NIST and Class (Survey)


\section{Big Data Stack (BDS)}
BDS is a collection of playbooks for deploying the tools of big data analytics, given a set of accessible IP addresses. It allows the user to select a subset of the available tools to deploy in order to customize their virtual cluster.
Example: to setup a stack with Hadoop, Spark, and Hive:
$ ansible-playbook play-hadoop.yml addons/spark.yml addons/hive.yml
Example: to setup a stack with Hadoop, HBase, and Drill:
$ ansible-playbook play-hadoop.yml addons/hbase.yml addons/drill.yml
It is provided online at https://github.com/futuresystems/big-data-stack

Roles and playbooks offered by BDS        (as of 27 Sep., 2016)
Roles
Drill
Ganglia
Hadoop
Hbase
Hive
Java
Limits
Maven
Mysql
Nagios
Pig

Roles (cont.)
Spark
Supervisord
Zookeeper

Playbooks
Hadoop
Hbase
Pig
Spark
Drill
Hive


\subsection{Towards Cloudmesh}
BDS is ideally suited to be included in cloudmesh as one of its deployment management tools. Cloudmesh adds features to easily access infrastructure.
  … more in the next another section we present later

Examples
Face Detection Example
Example 1: Face Detection Lifecycle
\begin{Verbatim}
[masters]
10.0.0.1
10.0.0.2
10.0.0.3
[workers]
10.0.0.4
10.0.0.5
10.0.0.6
Example 1: Face Detection Lifecycle on AWS
[masters]
10.0.0.1
10.0.0.2
10.0.0.3
[workers]
10.0.0.4
10.0.0.5
10.0.0.6
u'tag_masters_True': [u'52.23.213.103', u'54.196.41.145',u'54.209.137.235'],
u'tag_workers_True': [u'52.23.213.104', u'54.196.41.146',u'54.209.137.237'], ...
\end{Verbatim}

Example 1: Face Detection Screenshot Running EC2 provisioning playbook
on command line:

\begin{Verbatim}
$ ansible-playbook boot-ec2-with-vpc.yml -i ec2.py
\end{Verbatim}

\section{Virtual Clusters}
\subsection{Motivation}
Traditional HPC provided clusters to the community. The clusters are managed by professional staff and have mostly rigid software stack.
Advantages: provides well defined environment to the community, provides often optimized services for this particular system or the community it serves, allows sharing of resources through queuing system
Disadvantages: Different communities, Groups or projects may have different requirements that are not met by the software stack, software stack often not state-of-the art, but tuned for reliability, experimentation with new software and methodologies in such an environment is difficult, queuing system may not provide enough interactivity
Can we build a more flexible environment
Support custom software stacks
Support different interaction modes with instantaneous access without wait time
Support the use of heterogeneous platforms to allow increase in availability as well as features
Support the easy management of such an environment

\subsection{Virtual Clusters}

Definition: A set of compute, storage and network services that
interact with each other to serve an application or community for a
specific amount of time to support one or more experiments. A virtual
cluster is managed by the community or application user. They are
typically built on top of HPC, Grids, Clouds, and Containers. The
virtual cluster is managed by the user.

Contrasting Grids: targeted towards the support of a virtual
organization introducing high overheads on the deployment and
management of such infrastructure. Grids are a natural expansion of
traditional supercomputer centers that are managed by professional
staff. Contrasting Cloud IaaS: offer typically low level IaaS services
allowing users to combine them. They are a valuable building block for
virtual clusters. Clouds are managed by professional staff.

Contrasting Cloud PaaS: offer enhanced services to users targeting a
particular platform. They are offered by professional staff.

Contrasting Containers: offer an abstraction to share the existing hardware and OS while using containers making the it not necessary to us OS virtualization, thus saving space. Containers provide very useful enhancements for creating virtual clusters through add ons such as Kubernetes and Docker Swarm.



\subsection{Cloudmesh Access and Management of virtual clusters}
Cloudmesh is an API, command line tool, and command shell allowing the easy utilization of HPC, Clouds, Containers through machine abstractions
Switching between alternative services can be achieved by updating a
single variable.

Demonstrated usages of
Comet                                                                                           (HPC integrated Cloud)
FutureSystems, Chameleon, CloudLab, Bridges, Jetstream, …  (National Openstack Clouds)
Cybera (CA), Karslruhe Openstack Cloud (KIT, Germany), …     (International)
EC2, AWS, Azure                                                                          (Commercial Non Openstack)
Devstack, Trystack, Virtualbox                                                       (Desktop Clouds)
A user could use all of them

\subsection{Cloudmesh and Comet Cloud Virtual Cluster}
Comet is a NSF sponsored super computer offered by SDSC to the community. It operates in two modes: HPC and virtual clusters
Comet virtual clusters are low level and offer the HPC administrator the view of a cluster as if it were hardware. This is achieved via virtualization and low level exposure of network services. The administrator has full control over the cluster.
This is achieved by the integration of virtualization and SRIOV within Rocks. Thus the same cluster can not only be used in virtualized mode, but also in HPC mode
Comet users can therefore use HPC and virtualized clusters on the same hardware. Virtual clusters are treated as a special kind of compute job
Based on our long experience in this field we have delivered to SDSC an extension to cloudmesh that allows the creation and management of virtual clusters within comet
We can today use more than Comet
Accessing Virtual Clusters
REST API
Command line interface
Command shell for scripting
Console Access
Web Page

\begin{Verbatim}
Simple commands allow easy management
cm comet cluster  ID
Show the cluster details
cm comet power on ID vm-ID -[0-3] --walltime=6h
Power 3 nodes on for 6 hours
cm comet image attach image.iso ID vm-ID-0
Attach an image
cm comet boot ID vm-ID-0
Boot node 0
cm comet console vc4
Console 

Switching infrastructure in cloudmesh is easy
cm var cloud=jetstream / bridges / aws / …
cm default cloud=$cloud
cm boot
\end{Verbatim}

\subsection{Cloudmesh Launcher abstraction}
Important is that cloudmesh introduces an abstraction for creating virtual clusters that we call “launcher”. A launch specifies the creation of a virtual cluster. Such a launch can use
Ansible
Chef, …
Openstack Heat
Ssh
Make
Pegasus (Grids)
Hence a variety of different launch platforms can be integrated into cloudmesh and work from different community contributions can be supported.

Building virtual clusters is easy
cm launcher cluster -n 10
Gives a virtual cluster with 10 nodes and allows the user to login between each node


cm launcher hadoop -n 10 --cloud=rackspace
Finds the playbook for creating a hadoop cluster and launches it on a cloud called rackspace



Building enhance virtual clusters is easy
Virtual cluster for face detection
cm launcher facedetection -n 10 --cloud=rackspace
Finds the playbook for creating a virtual cluster and launches the deployment for the face detection project.
Cloudmesh will look in the git repository to locate the ansible scripts and execute them


Cloudmesh Launcher Portal (in development)
       Specific Launch                                    List of Launchers





Automatically generates Graphical representation of input parameters                                  
\subsection{Conclusion}
Ansible + Cloudmesh
Will enable re-usable specification of Big Data Stack applications in 87 use cases and 62 unique roles from which NIST has 27
BDS status
From this a total part of BDS are vetted (the most fundamental ones): 
14 Roles  (about 50% done of NIST roles)
6   Playbooks

\subsection{Cloudmesh Status}
We can replicate on different infrastructures including AWS, Azure, Openstack, Comet …
We need to complete the launcher and integration of the BDS into the launcher



