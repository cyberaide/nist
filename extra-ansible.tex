\section{Ansible}


\subsection{Why Ansible?}
Expandable: Leverages Python (default) but modules can be written in any language
Simplicity: 
Agentless: no setup required on managed node
Security: Allows deployment from user space; uses ssh for authentication
Flexibility: only requires ssh access to privileged user
Transparency: YAML Based script files express the steps of installing and configuring software
Modularity: Single Ansible Role (should) contain all required commands and variables to deploy software package independently
Sharing and portability: roles are available from source (github, bitbucket, gitlab, etc) or the Ansible Galaxy portal


\subsection{Features of Ansible fostering reproducibility}

Ansible provides a structured environment for defined sets of tasks to execute on a number of nodes. Tasks are ideally idempotent.
There are two basic methods for organizing tasks:
Playbooks: main entry points, specific to environments
Roles: reusable abstraction (analogous to functions)
Playbooks provide the primary entry point, and consist of tasks that are either defined directly in the playbook, or externally via roles.
Roles are groups of related tasks that can be distributed and reused independently of playbooks.
The Inventory file identifies the nodes controlled by Ansible.


\subsection{Ansible: Inventory, IaaS, PaaS}

The inventory file for Ansible groups the IP addresses of the infrastructure into named sets.
On the right are two examples of inventory files:
Top: a static inventory for a cluster with two nodes in the "webservers" group and one node in the "databases" group.
Bottom: a dynamic inventory in which a script is called to define the nodes. The output must be specifically-formatted JSON for Ansible.
\begin{Verbatim}
[webservers]
10.0.0.1
10.0.0.2
[databases]
10.0.0.3
#!/use/bin/env bash

python dynamic-nodes-setup.py
\end{Verbatim}

\subsection{Ansible: Roles}
Roles are primarily defined as a collection of related tasks to accomplish a single goal.  These tasks can be configured via variables, that may be overridden when the role is used in a playbook.
On the top right is an example directory structure defining a role for Apache Spark: Tasks are defined in the tasks/main.yml file, Variables are defined in defaults/main.yml
On bottom right is an elided set of the entries in tasks/main.yml. These tasks use the ``get\_url'', and ``unarchive'' modules and optionally builds from source when spark\_package\_type is "src"
\begin{Verbatim}
spark
- defaults
   - main.yml
- LICENSE
- README.md
- tasks
    - build.yml
    - main.yml
\end{Verbatim}

\begin{Verbatim}
---

- name: download
  get_url:
    url: "{{ spark_url }}"
    dest: "{{ ansible_env.PWD }}/{{ _spark_package_name }}"
    sha256sum: "{{ spark_sha256 }}"
  
- name: extract
  unarchive:
    dest: "{{ spark_install_dest }}"
    src: "{{ ansible_env.PWD }}/{{ _spark_package_name }}"
    copy: no
    creates: "{{ spark_install_dest }}/{{ _spark_directory }}"
  
- include: build.yml
  when: spark_package_type == 'src'

# ...

Ansible: Playbooks
An example playbook is shown to the right.

Name: description of this play

Hosts: the set of nodes on which to operate, named in the inventory file

Become: escalate privileges

Roles: list of roles to apply, with overridden variables as needed.
---

- name: install and configure spark
  hosts: frontendnodes
  become: yes
  roles:
    - role: java
    - role: maven
    - role: spark
      spark_version: 1.6.0
      spark_package_type: bin
\end{Verbatim}


\subsection{Ansible: Role Repositories}

There are two primary mechanisms available to redistribute roles

Ansible Galaxy is the primary hub for sharing roles. The
ansible-galaxy tool can install these roles. Useful for using
externally-developed roles

Directly from the source code repositories. In the case of using Git,
these repositories can be installed as git submodules. Useful for
internally-developed roles

We have found that installing roles as git submodules allows the
flexibility to both use roles developed elsewhere as well as make
modifications and even contribute those changes upstream.

\subsection{Ansible: Vetting Roles}
An important aspect of developing roles is the vetting step. In a
production environment, each role must be examined for security
vulnerabilities as well as to ensure that they will function as
expected. Externally-developed roles may need to be adapted to work in
the target environment. When a role is defined, variables to modify
the deployment behavior need to be exposed. We have found that many
roles made available only expose a subset of the application's
available configuration options and may not work outside the author's
environment.

Projects
List of Projects
27 roles are collected from 6 NIST use cases. Hadoop, Spark, HBase,
D3, Tableau, Java and Zookeeper are frequently reused more than four
out of six use cases.

Table 1. List of Projects with Roles

Table is in Google sheet: 
% \url{https://docs.google.com/spreadsheets/d/1idB11ibeITgTLUT9D_xT6cOMsVzdG2cVa3o0w_Wh7IE/edit?usp=sharing}

Roles from NIST and Class (Survey)
